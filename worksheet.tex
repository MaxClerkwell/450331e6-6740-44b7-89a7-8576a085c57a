\documentclass{dcbl/challenge}

\setdoctitle{Der Titel meines Dokuments}
\setdocauthor{Stephan Bökelmann}
\setdocemail{sboekelmann@ep1.rub.de}
\setdocinstitute{AG Physik der Hadronen und Kerne}


\begin{document}

When it comes to programming, we don't want our progress to be lost along the way. 
Even more, might want to get back to a certain point in time, before we deleted this wonderful function we once wrote, but after we applied several other changes.
In computer-games we usually have something like quicksaves. 
This is exactely what we will do in this worksheet.
In order for us, to set reproducable waypoints in our progress, we will use Git.
Git was originally developed by Linus Torvalds, to keep track of the progress of the Linux-Kernel.
Getting started with git can be confusing at first, but once mastered it can be a powerful tool for tracking changes and working with multiple people on the same project.

\section*{Exercises}
\begin{aufgabe}
    First of all, install Git using your Linux distribution's package manager (e.g., \texttt{apt-get install git} on Debian/Ubuntu or \texttt{yum install git} on Fedora).
    Configure Git with your name and email using \texttt{git config --global user.name} and \texttt{git config --global user.email}.
    Navigate into your home directory and create a new directory for our first project.
    The tool Git relies on a hidden folder, in the directory it should be tracking. 
    We can create this folder by running \texttt{git init}.
    \texttt{git init} creates a folder named \texttt{.git} in your project directory.
    All changes should be tracked in this folder.
    We can now use \texttt{git status} to check whether the folder is in a clean state or not.
\end{aufgabe}

\begin{aufgabe}        
    Create a new file named \texttt{README.md} in your repository directory.
    Add some content to the file, for example, a project title or description.
    Stage the file using \texttt{git add README.md}.
    Commit the changes with a message using \texttt{git commit -m "Initial commit"}.
\end{aufgabe}

\begin{aufgabe}
    Modify the \texttt{README.md} file by adding more information or making some changes.
    Stage and commit these changes.
    Use \texttt{git log} to view the commit history and note the commit hash of your first commit.
    Check out the first commit using \texttt{git checkout} followed by the first four characters of the commit hash.
    Explore the state of the repository at that point, then return to the latest state using \texttt{git checkout main} (or \texttt{master} if your default branch is named master).
\end{aufgabe}
\begin{aufgabe}
    Create a new branch named \texttt{feature} using \texttt{git branch feature}.
    Switch to the new branch with \texttt{git checkout feature}.
    Make changes to the \texttt{README.md} file or create a new file in the branch.
    Stage and commit the changes on this branch.
    Switch back to the main branch using \texttt{git checkout main}.
    Merge the changes from \texttt{feature} into \texttt{main} using \texttt{git merge feature}.
    
\end{aufgabe}
\begin{aufgabe}
    Make changes to any file in your repository, but do not stage or commit these changes.
    Use \texttt{git stash} to temporarily store the uncommitted changes.
    Check the status of your repository with \texttt{git status} to ensure it's clean.
    Apply the stashed changes back to your working directory with \texttt{git stash pop} and continue your work.

\end{aufgabe}

\section*{Annotations}
\begin{enumerate}
    \item Link zu einem YouTube-Video: \url{https://www.youtube.com}
\end{enumerate}

\end{document}
